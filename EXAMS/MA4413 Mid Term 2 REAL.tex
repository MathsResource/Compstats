 \documentclass[a4paper,12pt]{article}
%%%%%%%%%%%%%%%%%%%%%%%%%%%%%%%%%%%%%%%%%%%%%%%%%%%%%%%%%%%%%%%%%%%%%%%%%%%%%%%%%%%%%%%%%%%%%%%%%%%%%%%%%%%%%%%%%%%%%%%%%%%%%%%%%%%%%%%%%%%%%%%%%%%%%%%%%%%%%%%%%%%%%%%%%%%%%%%%%%%%%%%%%%%%%%%%%%%%%%%%%%%%%%%%%%%%%%%%%%%%%%%%%%%%%%%%%%%%%%%%%%%%%%%%%%%%
\usepackage{eurosym}
\usepackage{vmargin}
\usepackage{amsmath}
\usepackage{multicol}
\usepackage{graphics}
\usepackage{epsfig}
\usepackage{subfigure}
\usepackage{fancyhdr}

\setcounter{MaxMatrixCols}{10}
%TCIDATA{OutputFilter=LATEX.DLL}
%TCIDATA{Version=5.00.0.2570}
%TCIDATA{<META NAME="SaveForMode" CONTENT="1">}
%TCIDATA{LastRevised=Wednesday, February 23, 2011 13:24:34}
%TCIDATA{<META NAME="GraphicsSave" CONTENT="32">}
%TCIDATA{Language=American English}

\pagestyle{fancy}
\setmarginsrb{20mm}{0mm}{20mm}{25mm}{12mm}{11mm}{0mm}{11mm}
\lhead{MA4413 2013} \rhead{Mr. Kevin O'Brien}
\chead{Midterm Assessment 2 }
%\input{tcilatex}

\begin{document}
\subsection*{Part 1. Theory for Inference Procedures (4 Marks)}
Answer the four short questions. Each correct answer will be awarded 1 mark. Reasonably short answers will suffice, i.e. your answers should not exceed a paragraph.
\begin{itemize}
\item[(i)] (1 Mark) What is a $p-$value?
\item[(ii)] (1 Mark) Briefly describe how $p-$value is used in hypothesis testing
\item[(iii)] (1 Mark) What is meant by a Type I error?
\item[(iv)] (1 Mark) What is meant by a Type II error?
\end{itemize}
(Once you have completed this section, \textbf{please turn over.})
\newpage
\subsection*{Part 2. Normal Distribution (5 Marks)}
The strength of an analogue signal received at a detector (measured in microvolts : $\mu V$ ), is normally distributed with a mean of 100 $\mu V$  
and a variance of 100 $\mu V^2$ . 


\noindent With the signal strength denoted as $X$, we can say:

\[ X \sim N(100,100) \]

\begin{itemize}
\item[(i)] (1 Mark) What is the Z-score for 115 $\mu V$ ? 
\item[(ii)] (1 Mark) What is the Z-score for 80 $\mu V$ ?
\item[(iii)] (1 Mark) What is the probability that the signal will exceed 115 $\mu V$ ? 
\item[(iv)] (1 Mark) Estimate the value of $z_o$, which is an observation from the Standard Normal (Z) distribution, given that:
\[  P(Z \geq z_o) = 0.04.\] 
(Hint: you may use a reasonable close value)
\item[(v)] (1 Mark) What is the micro-voltage threshold $x_0$ below which 96\% of the signals will be? 
\[  P(X \leq x_o) = 0.96,\] 
\[  P(X \geq x_o) = 0.04.\]
%\item[(iii)] What is the micro-voltage below which 25\% of the signals will be?
%\item[(iv)](1 Mark) What is the probability that the signal will be less than 80 microvolts?
%\item[(v)] (1 Mark) What is the probability that the signal will be between 80 and 125 microvolts?
\end{itemize}
(Once you have completed this section, \textbf{please turn over.})
%\item[(iii)] What is the micro-voltage below which 25% of the signals will be? \item[(4 marks)]
\newpage
(\textbf{Room for Answers})
\newpage
\subsection*{Part 3. Confidence Interval (3 Marks)}

Suppose that the mean weight of a sample of 16 items is 160g, and the sample standard deviation is 32g.



\begin{itemize}
\item[(ii)] (1 Mark) Compute the standard error that would correspond to the sample mean.
\item[(ii)] (1 Mark) State the appropropriate quantile from the $t-$distribution that would be used to compute the 95\% confidence interval for the mean.
\item[(iii)] (1 Mark) Determine the 95\% confidence interval for the mean.
\end{itemize}
(Once you have completed this section, \textbf{please turn over.})
\newpage
\subsection*{Part 4. Hypothesis Testing (3 Marks)}

\noindent A sample of 500 voters was taken by a political pollster to estimate the proportion of first preference votes a particular candidate will obtain in a forthcoming election. \\
\bigskip

\noindent It was found that 280 out of these 500 voters would give the candidate their first preference.
\[\hat{p} = 56\%  \mbox{    (i.e.  } 0.56)\]

\vspace{0.4cm}
\noindent
Using a significance level of 5\%, test the hypothesis that the percentage of voters who will give this particular candidate their first preference in the election is 60\%.\\


\begin{itemize}
\item[(i)] (1 Mark) Formally state the null and alternative hypotheses. (You may work on the basis that this is a two-tailed hypothesis test.)

\item[(ii)] (1 Mark) Compute the Test Statistic for this hypothesis test.
\item[(iii)] (1 Mark) Given that the critical value is 1.96, state your conclusion for this test.
\end{itemize}
\newpage
(\textbf{Room for Answers})
\newpage


\section*{Formulas for Standard Errors}
\subsection*{Confidence Intervals}

{\bf One sample}
\begin{eqnarray*} S.E.(\bar{X})&=&\frac{\sigma}{\sqrt{n}}.\\\\
S.E.(\hat{P})&=&\sqrt{\frac{\hat{p}\times(100-\hat{p})}{n}}.\\
\end{eqnarray*}
\subsection*{Hypothesis tests}
{\bf One sample}
\begin{eqnarray*}
S.E.(\bar{X})&=&\frac{\sigma}{\sqrt{n}}.\\\\
S.E.(\pi)&=&\sqrt{\frac{\pi\times(100-\pi)}{n}}
\end{eqnarray*}
\end{document}
\[{\bf Two samples}
\begin{eqnarray*}
S.E.(\bar{X}_1-\bar{X}_2)&=&\sqrt{\frac{\sigma^2_1}{n_1}+\frac{\sigma_2^2}{n_2}}.\\\\
S.E.(\hat{P_1}-\hat{P_2})&=&\sqrt{\frac{\hat{p}_1\times(100-\hat{p}_1)}{n_1}+\frac{\hat{p}_2\times(100-\hat{p}_2)}{n_2}}.\\\\
\end{eqnarray*}
\]
\subsection*{Hypothesis tests}
\[{\bf One sample}
\begin{eqnarray*}
S.E.(\bar{X})&=&\frac{\sigma}{\sqrt{n}}.\\\\
S.E.(\pi)&=&\sqrt{\frac{\pi\times(100-\pi)}{n}}
\end{eqnarray*}
\]
\[{\bf Two large independent samples}
\begin{eqnarray*}
S.E.(\bar{X}_1-\bar{X}_2)&=&\sqrt{\frac{\sigma^2_1}{n_1}+\frac{\sigma_2^2}{n_2}}.\\\\
S.E.(\hat{P_1}-\hat{P_2})&=&\sqrt{\left(\bar{p}\times(100-\bar{p})\right)\left(\frac{1}{n_1}+\frac{1}{n_2}\right)}.\\
\end{eqnarray*}
\]

\end{document}