\documentclass[]{article}
\usepackage{framed}
\usepackage{amsmath}
\usepackage{amssymb}
\usepackage{multicol}
%opening

\begin{document}

%---------------------------------------%
\section*{Question 1 (Sample Variant 3)[25 marks]}

\begin{itemize}

\item[(a)] \textbf{\textit{Descriptive Statistics (5 Marks)}}\\
Consider the following data set of seven numbers:

\begin{center}
\textbf{\texttt{4, 18,  2,  7, 18,  9, 3, 11, 17, 19,  4 }}
\end{center}
% 4 Marks

\noindent For this sample, compute the following descriptive statistics (specifying the approach you have used):
\begin{itemize}
%\item[a.] (1 Mark) The median,
\item[(i)] (1 Mark) the Mean,
\item[(ii)] (1 Mark) the Median,
\item[(iii)] (2 Marks) the First and Third Quartiles,
\item[(iv)] (1 Mark) the Interquartile Range .
\end{itemize}

\item[(b)]\textbf{\textit{Independent Events (6 Marks)}}\\ Suppose A and B are two events, with P(A), the probability that A occurs, equal to 0.4 and P(B), the probability that B occurs, equal to 0.5. .
\begin{itemize}
\item[(i)] (2 Marks) Assume that A and B are independent events. Calculate P(A $\cap$ B), the probability of both A and B occuring. %if A and B are independent events.
\item[(ii)] (2 Marks) Assume that A and B are independent events. Calculate P(A $\cup$ B), the probability of either A or B (or both) occuring.
\item[(iii)] (1 Mark)  Assume that A and B are mutually exclusive events. Calculate P(A $\cap$ B), the probability of both A and B occuring. %if A and B are independent events.
\item[(iv)] (1 Mark) Assume that A and B are mutually exclusive events. Calculate P(A $\cup$ B), the probability of either A or B (or both) occuring.
%\item[i]               A and B are 
\end{itemize}


\item[(c)] \textbf{\textit{Probability (4 Marks)}}\\ An IT consultant is responsible for three software engineering projects X, Y and Z.
He knows that the probability of completing project X in time is 0.99, for project Y this probability is 0.95
and for project Z it is 0.80.

\begin{itemize}
\item[(i)] (1 Mark) What assumption do you need to make in order to calculate the probability
of completing all three projects in time, from the information given?
\item[(ii)] (3 Marks) Calculate the probability of completing all three projects in time.
%\item[(iii)] (2 marks) Calculate the probability that only projects X and Y will be completed on time.
\end{itemize}
\newpage
\item[(d)] \textbf{\textit{Probability (5 Marks)}}\\ The following contingency table illustrates the number of 400 students in different
departments according to gender.

\begin{center}
\begin{tabular}{|c|c|c|c|}
  \hline
  % after \\: \hline or \cline{col1-col2} \cline{col3-col4} ...
   & Computer Science & Statistics & Equine Science \\\hline
  Males & 140 & 100 & 20  \\  \hline
  Females & 30 & 80 & 30  \\ \hline

  \hline
\end{tabular}
\end{center}

\begin{itemize}
\item[(i)] (2 Marks) What is the probability that a randomly chosen person from the sample is a
computer science student?
\item[(ii)] (2 Marks) What is the probability that a randomly chosen person from the sample is both female and studying statistics?
%\item[(iii)] (2 marks) What is the probability that a randomly chosen person from the sample is male?
%\item[(iv)] (2 marks) Given that a student studies statistics, what is the probability that the student is female?
%\item[(iii)] (2 marks) What is the probability that a randomly chosen person from the sample is a
%male or a statistics student?
\item[(iii)] (1 Marks) Given that the student is female, what is the probability that she is an
equine science student?
\end{itemize}
\item[(e)] \textbf{\textit{Discrete Random Variables (3 Marks)}}\\The probability distribution of discrete random variable $X$ is tabulated below. There are 6 possible outcome of $X$, i.e. 1, 2, 3, 4 ,5 and 6.
\begin{center}
\begin{tabular}{|c||c|c|c|c|c|c|}
\hline
$x_i$  & 1 & 2 & 3 & 4 & 5 & 6 \\\hline
$P(x_i)$ & 0.16 & 0.13 & \mbox{   k   } & 0.19 & 0.21 & 0.12\\
\hline
\end{tabular}
\end{center}

\begin{itemize}
\item[(i)] (1 Marks) Compute the value for $k$.
\item[(ii)] (2 Marks) Determine the expected value $E(X)$.
%\item[iii] (2 marks) Evaluate $E(X^2)$.
%\item[iv] (1 marks) Compute the variance of random variable $X$.
\end{itemize}

\item[(f)] \textbf{\textit{Binomial Coefficients (2 Marks)}}\\ Evaluate the following binomial coefficients
\[  { 6 \choose 4}   \mbox{  and }  { 6 \choose 0}\] 
\end{itemize}
\end{document}