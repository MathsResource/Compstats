\documentclass{beamer}

\usepackage{amsmath}
\usepackage{amssymb}

\begin{document}
%======================================================================= %
\begin{frame}
	\Large
\frametitle{Huffman Coding}
\begin{itemize}
\item[1.] Constructing a frequency table sorted in descending order. \bigskip
\item[2.] Building a binary tree\\ 
Carrying out iterations until completion of a complete binary tree:{\Large

	\item[   (a)]   Merge the last two items (which have the minimum frequencies) of
the frequency table to form a new combined item with a sum
frequency of the two.
\item[(b)]    Insert the combined item and update the frequency table.

}
\end{itemize}
\end{frame}
%======================================================================= %
\begin{frame}
	\Large
\frametitle{Huffman Coding}
\begin{itemize}
\item[3.] Deriving Huffman tree\\
Starting at the root, trace down to every leaf; mark `0' for a left branch
and `1' for a right branch.
\item[4.] Generating Huffman code:\\
Collecting the 0s and 1s for each path from the root to a leaf and
assigning a 0-1 codeword for each symbol.
\end{itemize}
\end{frame}
%======================================================================= %
%======================================================================= %
\begin{frame}
\end{frame}
\end{document}