 \documentclass[a4paper,12pt]{article}
%%%%%%%%%%%%%%%%%%%%%%%%%%%%%%%%%%%%%%%%%%%%%%%%%%%%%%%%%%%%%%%%%%%%%%%%%%%%%%%%%%%%%%%%%%%%%%%%%%%%%%%%%%%%%%%%%%%%%%%%%%%%%%%%%%%%%%%%%%%%%%%%%%%%%%%%%%%%%%%%%%%%%%%%%%%%%%%%%%%%%%%%%%%%%%%%%%%%%%%%%%%%%%%%%%%%%%%%%%%%%%%%%%%%%%%%%%%%%%%%%%%%%%%%%%%%
\usepackage{eurosym}
\usepackage{vmargin}
\usepackage{amsmath}
\usepackage{multicol}
\usepackage{graphics}
\usepackage{epsfig}
\usepackage{subfigure}
\usepackage{fancyhdr}

\setcounter{MaxMatrixCols}{10}
%TCIDATA{OutputFilter=LATEX.DLL}
%TCIDATA{Version=5.00.0.2570}
%TCIDATA{<META NAME="SaveForMode" CONTENT="1">}
%TCIDATA{LastRevised=Wednesday, February 23, 2011 13:24:34}
%TCIDATA{<META NAME="GraphicsSave" CONTENT="32">}
%TCIDATA{Language=American English}

\pagestyle{fancy}
\setmarginsrb{20mm}{0mm}{20mm}{25mm}{12mm}{11mm}{0mm}{11mm}
\lhead{MA4413 2013} \rhead{Mr. Kevin O'Brien}
\chead{Midterm Assessment 1 }
%\input{tcilatex}

\begin{document}

\section*{Attempt ALL questions}

\subsection*{Question 1 Probability [2 Marks]}
%-----------------------------------%

\textbf{Description of the Experiment}
\begin{itemize}
\item Suppose we have two bags, each containing 2 marbles. 
\item One bag has 2 red marbles and
the other has a red marble and a yellow marble. 
You pick a bag at random and then pick
one of the marbles in that bag at random. 
\item When you look at the marble, it is red. 
You
now pick the second marble from that same bag.
\end{itemize} 
\textbf{Questions}
\begin{itemize}
\item[(a)](1 Mark)  Write down the sample space for this experiment, where the outcomes are the ordered pairs of drawn marbles.
\item[(b)](1 Mark) What is the probability that this marble
is also red? Select one of the options below, with a justification for your answer.
You may justify your answer by references to sample points.
\end{itemize}
\textbf{Options for part b}
\begin{center}
\begin{multicols}{4}
\begin{itemize}
\item[(i)] 1/4
\item[(ii)] 1/3
\item[(iii)] 2/3
\item[(iv)] 1/2
\end{itemize}
\end{multicols}
\end{center}


\bigskip
%-----------------------------------%
\subsection*{Question 2 Descriptive Statistics [3 Marks]}

Consider the following data set of seven numbers:

\begin{center}
\textbf{\texttt{18, 14,  8, 15, 17,  9, 10 }}
\end{center}
% 4 Marks

\noindent For this sample, compute the following descriptive statistics:
\begin{itemize}
%\item[a.] (1 Mark) The median,
\item[(a)] (1 Mark) The mean,
\item[(b)] (1 Mark) The median,
\item[(c)] (1 Mark) The standard deviation.
\end{itemize}

\newpage
%Progress 7/15
\subsection*{Question 3 Discrete Random Variable [2 Marks]}
The probability distribute of discrete random variable $X$ is tabulated below. There are 5 possible outcome of $X$, i.e. 1, 2, 3, 4 and 5.
{
\large
\begin{center}
\begin{tabular}{|c||c|c|c|c|c|}
\hline
$x_i$  & 1 & 2 & 3 & 4 & 5  \\\hline
$p(x_i)$ & 0.30 & 0.20 & 0.20 & 0.10 & 0.20 \\
\hline
\end{tabular}
\end{center}
}
\begin{itemize}
%\item[a.] (1 Mark) Compute the value of $k$.
\item[(a)] (1 Mark) What is the expected value of X?
\item[(b)] (1 Mark) Compute the value of $E(X^2)$,
\item[(c)] (1 Mark) Compute the variance of $X$.
\end{itemize}
\bigskip
\section*{Question 4 Poisson Distribution [3 Marks]}

\begin{itemize}
\item Suppose that, for a Poisson random variables $X$, the expected number of occurences in a one hour time period is 2. 


\item Furthermore, the probability of exactly one occurence in a one
 hour time period is $0.2706$.

\item Based on this information, answer the following questions.
\end{itemize}
\noindent{\textbf{Questions}}
\begin{itemize}
\item[(a)] (1 Mark) Compute the probability of no occurences in a one hour period?
\item[(b)] (1 Mark) What is the probability of two or more occurences in a one hour period?
\item[(c)] (1 Mark) Suppose $Y$ is a random variable that describes the amount of time between occurences. What probability distribution should be used to describe $Y$?
\end{itemize}
%Progress 9/15


\noindent (When answering, justify your answer with workings, or by reference to an
axiom, theorem or rule.)

\newpage

%-----------------------------------%
\subsection*{Question 5 Binomial Distribution [3 Marks] } % 12 Marks
A biased coin yields `Tails' on $45\%$ of throws. Consider an experiment that consists of throwing this coin 5 times.
\begin{itemize}
\item[(a)] (1 Mark) Evaluate the following term $^{5}C_3$.
\item[(b)] (1 Mark) Compute the probability of getting three `Tails' in this experiment.
\item[(c)] (1 Mark) Compute the probability of getting two `Heads' in this experiment.
\end{itemize}

\noindent(When answering, justify your answer with workings, or by reference to an
axiom, theorem or rule.)

\bigskip






%-----------------------------------%
\subsection*{Question 6 Exponential Distrbution [2 Marks]}
% Exponential [3 Marks]

Suppose an exponential random variable $X$ describes a ``lifetime" denominated in days, has a rate parameter of $\lambda = 1/7$
\begin{itemize}
\item[(a)] (1 Mark) What is the probability of a lifetime less than 5 days?
\item[(b)] (1 Mark) What is the probability of lifetime greater than 7 days?
\end{itemize}


\noindent(When answering, justify your answer with workings, or by reference to an
axiom, theorem or rule.)
\bigskip






%\subsection*{Q6. Poisson Distribution (2 Marks) }  % 14 Marks
%Suppose that a telephone help-line receives 4 calls per hour during offices hours.
%\begin{itemize}
%\item[a.] (1 Mark) Compute the value of $m$ for a 30 minute period during office hours.
%\item[b.] (1 Mark) Compute the probability of the help-line getting exactly one call in a 30 minute period during office hours.
%\end{itemize}
%\bigskip
%\subsection*{Q7. Exponential Distribution (1 Mark)} % 15 Marks


%\begin{itemize}

%\item[a.] (1 Mark) Compute the value of $P(X \leq 482)$

%\end{itemize}
\newpage
\section*{Formulae}
\subsection*{Descriptive Statistics}
\begin{itemize}
\item Sample Variance
\begin{equation*}
s^2 = \frac{\sum^{n}_{i=i} (x_i-\bar{x})^2}{n-1}
\end{equation*}
\end{itemize}
%-------------------------------------------------%
\subsection*{Probability}
\begin{itemize}

\item Conditional probability:
\begin{equation*}
P(B|A)=\frac{P\left( A\text{ and }B\right) }{P\left( A\right) }
\end{equation*}


\item Bayes' Theorem:
\begin{equation*}
P(B|A)=\frac{P\left(A|B\right) \times P(B) }{P\left( A\right) }
\end{equation*}





\item Binomial probability distribution:
\begin{equation*}
P(X = k) = \text{  }^{n}C_{k} \times p^{k} \times \left( 1-p\right) ^{n-k}\qquad \left( \text{where  }
^{n}C_{k} =\frac{n!}{k!\left(n-k\right) !} \right)
\end{equation*}

\item Poisson probability distribution:
\begin{equation*}
P(X = k) =\frac{m^{k}\mathrm{e}^{-m}}{k!}
\end{equation*}

\item Exponential probability distribution:
\begin{equation*}
P(X \leq k) = \begin{cases}
1-e^{- k/\mu}, & k \ge 0, \\
0, & k < 0.
\end{cases}\qquad \left( \text{where  }
\mu = {1\over \lambda}\right)
\end{equation*}
\end{itemize}


\end{document} 