\documentclass[a4]{beamer}
\usepackage{amssymb}
\usepackage{graphicx}
\usepackage{subfigure}
\usepackage{newlfont}
\usepackage{amsmath,amsthm,amsfonts}
%\usepackage{beamerthemesplit}
\usepackage{pgf,pgfarrows,pgfnodes,pgfautomata,pgfheaps,pgfshade}
\usepackage{mathptmx} % Font Family
\usepackage{helvet} % Font Family
\usepackage{color}
\mode<presentation> {
	\usetheme{Default} % was Frankfurt
	\useinnertheme{rounded}
	\useoutertheme{infolines}
	\usefonttheme{serif}
	%\usecolortheme{wolverine}
	% \usecolortheme{rose}
	\usefonttheme{structurebold}
}
\setbeamercovered{dynamic}
\title[MA4413]{Statistics for Computing \\ {\normalsize MA4413 Lecture 11A}}
\author[Kevin O'Brien]{Kevin O'Brien \\ {\scriptsize kevin.obrien@ul.ie}}
\date{Autumn 2011}
%\institute[Maths \& Stats]{Dept. of Mathematics \& Statistics, \\ University \textit{of} Limerick}
\renewcommand{\arraystretch}{1.5}
%------------------------------------------------------------------------%

\begin{document}

\frame{
\frametitle{Huffman encoding algorithm}

A frequency based coding scheme (algorithm) that follows Huffman�s idea is called Huffman coding. Huffman coding is a simple algorithm that generates a set of variable-size codewords of the minimum average length. The algorithm for Huffman encoding involves the following steps:
}
%----------------------------------------------------------------------------------%

\frame{
\begin{itemize}
\item[1.] Frequency Table: Constructing a frequency table sorted in descending order.

\item[2.] Building a binary tree:
    Carrying out iterations until completion of a complete binary tree:
    \begin{itemize}
    \item[(a)] Merge the last two items (which have the minimum frequencies) of    the frequency table to form a new combined item with a sum
    frequency of the two.
    \item[(b)] Insert the combined item and update the frequency table.
    \end{itemize}

\item[3.] Deriving Huffman tree:
Starting at the root, trace down to every leaf; mark �0� for a left branch and �1� for a right branch.

\item[4.] Generating Huffman code:
Collecting the 0s and 1s for each path from the root to a leaf and assigning a 0-1 codeword for each symbol.

\end{itemize}
}
%----------------------------------------------------------------------------------%

\frame{
\frametitle{Huffman Coding}
Huffman coding is a method of lossless data compression, and a form of entropy encoding. The basic idea is to map an alphabet to a representation for that alphabet, composed of strings of variable size, so that symbols that have a higher probability of occurring have a smaller representation than those that occur less often.

}
%----------------------------------------------------------------------------------%

\frame{
\frametitle{Huffman Coding}
The key to Huffman coding is Huffman's algorithm, which constructs an extended binary tree of minimum weighted path length from a list of weights. For this problem, our list of weights consists of the probabilities of symbol occurrence. From this tree (which we will call a Huffman tree for convenience), the mapping to our variable-sized representations can be defined.
}
%-------------------------------------------------------------------------------------------%
\frame{
\frametitle{Huffman Coding}
The mapping is obtained by the path from the root of the Huffman tree to the leaf associated with a symbol's weight. The method can be arbitrary, but typically a value of 0 is associated with an edge to any left child and a value of 1 with an edge to any right child (or vice-versa). By concatenating the labels associated with the edges that make up the path from the root to a leaf, we get a binary string. Thus the mapping is defined.
}


\end{document}
%-------------------------------------------------------------------------------------------%
\end{document} 