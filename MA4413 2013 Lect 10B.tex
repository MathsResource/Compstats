\documentclass[a4]{beamer}
\usepackage{amssymb}
\usepackage{graphicx}
\usepackage{subfigure}
\usepackage{newlfont}
\usepackage{amsmath,amsthm,amsfonts}
%\usepackage{beamerthemesplit}
\usepackage{pgf,pgfarrows,pgfnodes,pgfautomata,pgfheaps,pgfshade}
\usepackage{mathptmx}  % Font Family
\usepackage{helvet}   % Font Family
\usepackage{color}

\mode<presentation> {
 \usetheme{Default} % was
 \useinnertheme{rounded}
 \useoutertheme{infolines}
 \usefonttheme{serif}
 %\usecolortheme{wolverine}
% \usecolortheme{rose}
\usefonttheme{structurebold}
}

\setbeamercovered{dynamic}

\title[MA4413]{Statistics for Computing \\ {\normalsize MA4413 Lecture 10B}}
\author[Kevin O'Brien]{Kevin O'Brien \\ {\scriptsize Kevin.obrien@ul.ie}}
\date{Autumn Semester 2013}
\institute[Maths \& Stats]{Dept. of Mathematics \& Statistics, \\ University \textit{of} Limerick}

\renewcommand{\arraystretch}{1.5}


\begin{document}


\begin{frame}
\titlepage
\end{frame}

%---------%
\begin{frame}
\frametitle{Using \texttt{R} for Inference Procedures}
\begin{itemize}
\item[1] Review of the Paired t-test.
\item[2] Paired t-test using \texttt{R}
%\item[3] Two Sample Test for Proportions
\item[3] Test for the equality of variances for two samples
\item[4] Shapiro-Wilk Test for Normality
\item[5] Graphical procedures for assessing normality
\item[6] Grubb's Procedure for Determinin an Outlier
\end{itemize}
\end{frame}
%---------%
\begin{frame}
\frametitle{p-values using \texttt{R}}
\begin{itemize}
\item In every inference procedure performed using \texttt{R}, a p-value is presented to the screen for the user to interpret.

\item If the p-value is larger than a specified threshold $\alpha/k$ then the appropriate conclusion is a
failure to reject the null hypothesis.

\item Conversely, if the p-value is less than threshold, the appropriate conclusion is to reject the null hypothesis.

\item In this module, we will use a significance level$\alpha=0.05$ and almost always the procedures will be two tailed ($k=2$). Therefore the threshold $\alpha/k$ will be $0.025$.
\end{itemize}
\end{frame}

%---------%
\begin{frame}
\frametitle{Using Confidence Limits}
\begin{itemize}
\item Alternatively, we can use the confidence interval to make a decision on whether or not we should reject or fail to reject the null hypothesis.
\item If the null value is within the range of the confidence limits, we fail to reject the null hypothesis.
\item If the null value is outside the range of the confidence limits, we reject the null hypothesis.
\item Occasionally a conclusion based on this approach may differ from a conclusion based on the p-value. In such a case, remark upon this discrepancy.
\end{itemize}
\end{frame}


%-------------------------------------------------%

