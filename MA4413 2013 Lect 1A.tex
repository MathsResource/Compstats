\documentclass[a4]{beamer}
\usepackage{amssymb}
\usepackage{graphicx}
\usepackage{subfigure}
%\usepackage{newlfont}
\usepackage{amsmath,amsthm,amsfonts}
%\usepackage{beamerthemesplit}
\usepackage{pgf,pgfarrows,pgfnodes,pgfautomata,pgfheaps,pgfshade}
\usepackage{mathptmx}  % Font Family
%\usepackage{helvet}   % Font Family
\usepackage{color}

\mode<presentation> {
 \usetheme{Default} % was
 \useinnertheme{rounded}
 \useoutertheme{infolines}
 \usefonttheme{serif}
 %\usecolortheme{wolverine}
% \usecolortheme{rose}
\usefonttheme{structurebold}
}

\setbeamercovered{dynamic}

\title[MA4413]{Statistics for Computing \\ {\normalsize Lecture 1A}}
\author[Kevin O'Brien]{Kevin O'Brien \\ {\scriptsize Kevin.obrien@ul.ie}}
\date{Autumn Semester 2013}
\institute[Maths \& Stats]{Dept. of Mathematics \& Statistics, \\ University \textit{of} Limerick}

\renewcommand{\arraystretch}{1.5}

\begin{document}


\begin{frame}
\titlepage
\end{frame}

\frame{
\frametitle{About this module}

\begin{itemize}
\item Lecture 1: Monday 16-17 in Room P1033
\item Lecture 2: Wednesday 12-13 in Room FB028
\item No classes on Bank Holiday Monday and during the open days.
\item Tutorials - Tuesdays and Friday.\\
(You may attend any tutorial you like.)
\item Tutorials will take place from Week 2 to Week 13 inclusive.
\item Review classes will take place during week 13 if requested.
\end{itemize}

}
%-------------------------------------------------------%
\frame{
\frametitle{Today's Class}

\begin{itemize}
\item Description of module syllabus
\item Details on the assessment
\item Teaching materials
\begin{itemize}
\item Lecture notes : Website (bit.ly/ma4413) and available on SULIS system.
\item \texttt{R} code, a statistical programming language, will also be included in some parts of the course.
\item My e-mail address is on the lecture slides.
\end{itemize}
\end{itemize}

}
%-------------------------------------------------------%
\frame{
\frametitle{Syllabus}
The syllabus is made up of the following subject areas:
\begin{itemize}
\item Probability theory,
\item Descriptive statistics and graphical methods,
\item Probability distributions,\\
\item Information theory and data compression,
\item Confidence intervals and hypothesis testing.
\end{itemize}
}
%-------------------------------------------------------%
\frame{
\frametitle{Learning Outcomes}
On successful completion of this module, students should be able to:
\begin{itemize}
\item[1] Apply probability theory to problem solving.
\item[2] Employ the concepts of random variables and probability distributions to
problem solving.
\item[3] Apply information theory to solve problems in data compression and
transmission.
\item[4] Analyse rates and proportions.
\item[5] Perform hypothesis tests for a variety of statistical problems.
\end{itemize}

}
%-------------------------------------------------------%
\frame{
\frametitle{Module Assessment}
The grading of the module is as follows:
\begin{itemize}
\item End of semester examination ($70\%$).
\item Two mid-term examinations ($15\%$ each).
\item The mid-terms are provisionally scheduled for Week 5 and Week 9.
\item I will re-arrange the date if needs be.
\item Sample papers and past papers will be made available through the SULIS system or by email.
\end{itemize}
}
%-------------------------------------------------------%

\frame{
\frametitle{Probability}
\textbf{Introduction to Probability}\\
\textit{Probability is the very guide of life} -  Cicero
\begin{itemize}
\item[1] Probability
\item[2] Random Experiment
\item[3] Outcome
\item[4] Sample Space
\item[5] Event
\item[6] Conditional Probability
\item[7] Independent Events
\item[8] Mutually Exclusive Events
\item[9] Addition Rule
\end{itemize}
}
%------------------------------------------------------------%

\frame{
\frametitle{Probability}
\begin{itemize}
\item Probability theory is the mathematical study of randomness. A
probability model of a random experiment is defined by assigning
probabilities to all the different outcomes.
\item Probability is a numerical measure of the likelihood that an event will
occur. Thus, probabilities can be used as measures of degree of
uncertainty associated with outcomes of an experiment.
Probability values are always assigned on a scale from 0 to 1.
\item A probability of 0 means that the event is impossible, while
a probability near 0 means that it is highly unlikely to occur.
\item Similarly an event with probability 1 is certain to occur, whereas an
event with a probability near to 1 is very likely to occur.
\end{itemize}

}
%--------------------------------------------------------------------------------%
\frame{
\frametitle{Experiments and Outcomes}
\begin{itemize}
\item In the study of probability any process of observation is referred to as an
experiment.
\item The results of an experiment (or other situation involving uncertainty)
are called the outcomes of the experiment.
\item An experiment is called a random experiment if the outcome can not be
predicted.
\item Typical examples of a random experiment are
\begin{itemize}
\item a role of a die,
\item a toss of a coin,
\item drawing a card from a deck.
\end{itemize}If the experiment is yet to be performed we refer to �possible outcomes�
or �possibilities� for short. If the experiment has been performed, we
refer to �realized outcomes� or �realizations�.
\end{itemize}
}

%--------------------------------------------------------------------------------%
\frame{
\frametitle{Sample Spaces and Events}

\begin{itemize}
\item The set of all possible outcomes of a probability experiment is called a
\textbf{\emph{sample space}}, which is usually denoted by \textbf{\emph{S}}.
\item The sample space is an exhaustive list of all the possible outcomes of an
experiment. We call individual elements of this list \textbf{\emph{sample points}}.
\item Each possible outcome is represented by one and only one sample point
in the sample space.
\end{itemize}
}

%--------------------------------------------------------------------------------%
\frame{
\frametitle{Sample Spaces: Examples}
For each of the following experiments, write out the sample space.
\begin{itemize}
\item Experiment: Rolling a die once
\begin{itemize}
\item Sample space $S = \{1,2,3,4,5,6\}$
\end{itemize}
\item Experiment: Tossing a coin
\begin{itemize}
\item Sample space $S = \{ Heads , Tails\}$
\end{itemize}
\item Experiment: Measuring a randomly selected person�s height (cms)
\begin{itemize}
\item Sample space $S =$ The set of all possible real numbers.
\end{itemize}
\end{itemize}
}
%--------------------------------------------------------------------------------%
\frame{
\frametitle{Events}

\begin{itemize} \item An event is a specific outcome, or any collection of outcomes of an
experiment.
\item Formally, any subset of the sample space is an event.
\item Any event which consists of a single outcome in the sample space is
called an \textbf{\emph{elementary}} or \textbf{\emph{simple event}}.
\item Events which consist of more than one outcome are called \textbf{\emph{compound
events.}}
\item For example, an elementary event associated with the die example could
be the ``die shows 3".
\item An compound event associated with the die example could be the ``die
shows an even number".
\end{itemize}
}
%--------------------------------------------------------------------------------%
\frame{
\frametitle{The Complement Event}

\begin{itemize} 

\item The complement of an event $A$ is the set of all outcomes in the sample
space that are not included in the outcomes of event $A$.
\item We call the complement event of $A$ as $A^c$.
\item The complement event of a die throw resulting in an even number is the
die throwing an odd number.
\item Question: if there is a $40\%$ chance of a randomly selected student being male, what is the probability of the selected student being female?
\end{itemize}
}

%--------------------------------------------------------------------------------%
\frame{
\frametitle{Set Theory : Union and Intersection}

Set theory is used to represent relationships among events.\\ \bigskip

\noindent \textbf{Union of two events:}\\
The union of events A and B is the event containing all the sample points
belonging to A or B or both. This is denoted $A\cup B$, (pronounce as ``A union
B").\\ \bigskip
\noindent \textbf{Intersection of two events:}\\
The intersection of events A and B is the event containing all the sample
points common to both A and B. This is denoted $A\cap B$, (pronounce as ``A intersection
B").
}

%--------------------------------------------------------------------------------%
\frame{
\frametitle{More Set Theory}

In general, if A and B are two events in the sample space S, then
\begin{itemize} 
\item $A \subseteq B$ (A is a subset of B) = `if A occurs, so does B�
\item $\varnothing$ (the empty set) = an impossible event
\item $S$ (the sample space) = an event that is certain to occur
\end{itemize}
}

%--------------------------------------------------------------------------------%
\frame{
\frametitle{Examples of Events}

Consider the experiment of rolling a die once. From before, the sample space
is given as $S = \{ 1,2,3,4,5,6\}$. The following are examples of possible events.
\begin{itemize} 
\item A = score $< 4$ = $\{ 1,2,3\}$.
\item B = `score is even' = $\{ 2,4,6\}$.
\item C = `score is 7' = 0
\item $A\cup B$ = `the score is $< 4$  or even or both' = $\{ 1,2,3,4,6\}$
\item $A\cap B$ = `the score is $< 4$  and even� = $ \{ 2 \}$
\item $A^c$ =`event A does not occur' = $ \{ 4,5,6\}$
\end{itemize}
}




%--------------------------------------------------------------------------------%
\frame{
\frametitle{Probability}
If there are n possible outcomes to an experiment, and m ways in which event
A can happen, then the probability of event A ( which we write as P(A)) is
\[ P(A) = \frac{m}{n} \]

The probability of the event A may be interpreted as the proportion of times
that event A will occur if we repeat the random experiment an infinite number
of times.\\ \bigskip

\textbf{Rules}:\\
\begin{itemize}
\item[1] $0 \leq P(A) \leq 1 $: the probability of any event lies between 0 and 1
inclusive.
\item[2] $P(S) = 1$ : the probability of the sample space is always equal to 1.
\item[3] $P(A^c) = 1-P(A)$ : how to compute the probability of the complement.
\end{itemize}
}
%--------------------------------------------------------------------------------%
\frame{
\frametitle{Conditional Probability}
Suppose $B$ is an event in a sample space $S$ with $P(B) > 0$.
The probability that an event $A$ occurs once $B$has occurred or, specifically, the
conditional probability of A given $B$ (written $P(A|B)$), is defined as follows:
\[ P(A|B) = \frac{P(A\cap B)}{P(B)}\]

\begin{itemize}
\item This can be expressed as a multiplication theorem
\[ P(A\cap B) = P(A|B)\times P(B) \]
\item The symbol $|$ is a vertical line and does not imply division.
\item Also $P(A|B)$ is not the same as $P(B|A)$.
\end{itemize}
Remark: The Prosecutor's Fallacy , with reference to the O.J. Simpson trial.
}

%--------------------------------------------------------------------------------%
\frame{
\frametitle{Independent Events}
Events A and B in a probability space $S$ are said to be independent if the
occurrence of one of them does not influence the occurrence of the other.\\ \bigskip

More specifically, $B$ is independent of A if $P(B)$ is the same as $P(B|A)$. Now
substituting $P(B)$ for $P(B|A)$ in the multiplication theorem from the previous
slide yields.
\[ P(A\cap B) = P(A)\times P(B)\]
We formally use the above equation as our definition of independence.

}
%--------------------------------------------------------------------------------%
\frame{
\frametitle{Mutually Exclusive Events}
\begin{itemize}
\item Two events are mutually exclusive (or disjoint) if it is impossible for
them to occur together.
\item Formally, two events $A$ and $B$ are mutually exclusive if and only if
$A\cap B$ = $\varnothing$ \end{itemize}\bigskip
Consider our die example
\begin{itemize} 
\item Event A = `observe an odd number' = $\{1,3,5\}$
\item Event B = `observe an even number' = $\{2,4,6\}$

\item $A\cap B$ = $\varnothing$ (i.e. the empty set), so $A$ and $B$ are mutually exclusive.
\end{itemize}
}

%--------------------------------------------------------------------------------%
\frame{
\frametitle{Addition Rule}
The addition rule is a result used to determine the probability that event $A$ or
event $B$ occurs or both occur. The result is often written as follows, using set
notation:
\[ P(A\cup B) = P(A) + P(B)- P(A \cap B)\] 
\begin{itemize}
\item $P(A)$ = probability that event $A$ occurs.
\item $P(B)$ = probability that event $B$ occurs.
\item $P(A\cup B)$ = probability that either event $A$ or event $B$ occurs, or both
occur.
\item $P(A\cap B)$ = probability that event $A$ and event $B$ both occur.
\end{itemize}\bigskip

\noindent \textbf{Remark:} $P(A\cap B)$ is subtracted to prevent the relevant outcomes being
counted twice.


}

%--------------------------------------------------------------------------------%
\frame{
\frametitle{Addition Rule (Continued)}
For mutually exclusive events, that is events which cannot occur together:
$P(A\cap B) = 0$. The addition rule therefore reduces to
\[ P(A\cup B) = P(A) + P(B)\]
}
%--------------------------------------------------------------------------------%
\frame{
\frametitle{Addition Rule: Worked Example}
Suppose we wish to find the probability of drawing either a Queen or a Heart
in a single draw from a pack of 52 playing cards. We define the events $Q$ =
`draw a queen' and $H$ = `draw a heart'.
\begin{itemize}
\item $P(Q)$ probability that a random selected card is a Queen
\item  $P(H)$ probability that a randomly selected card is a Heart.
\item  $P(Q\cap H)$ probability that a randomly selected card is the Queen of
Hearts.
\item  $P(Q\cup H)$ probability that a randomly selected card is a Queen or a Heart.
\end{itemize}
}
%--------------------------------------------------------------------------------%
\frame{
\frametitle{Solution}
\begin{itemize} 
\item Since there are 4 Queens in the pack and 13 Hearts, so the $P(Q)$ = $4/52$
and $P(H) = 13/52$ respectively.
\item The probability of selecting the Queen of Hearts is $P(Q\cap H) = 1/52$.
\item We use the addition rule to find $P(Q\cup H)$:
\[ P(Q\cup H) = (4/52) + (13/52) - (1/52) = 16/52 \]
\item So, the probability of drawing either a queen or a heart is
$16/52 (= 4/13)$.
\end{itemize}
}
%--------------------------------------------------------------------------------%

\frame{
\frametitle{More on probability}
For this lecture and the next.
\begin{enumerate}
\item Contingency Tables
\item Conditional Probability: Worked Examples
\item Joint Probability Tables
\item The Multiplication Rule
\item Law of Total Probability
\item Bayes' Theorem
\item Exam standard Probability Question
\item Random Variables
\end{enumerate}

}
%-------------------------------------------------------%
\frame{
\frametitle{Contingency Tables}
Suppose there are 100 students in a first year college intake.  \begin{itemize} \item 44 are male and are studying computer science, \item 18 are male and studying engineering \item 16 are female and studying computer science, \item 22 are female and studying engineering. \end{itemize}

We assign the names $M$, $F$, $C$ and $E$ to the events that a student, randomly selected from this group, is male, female, studying computer science, and studying engineering respectively.
}
%-------------------------------------------------------%
\frame{
\frametitle{Contingency Tables}
The most effective way to handle this data is to draw up a table. We call this a \textbf{\emph{contingency table}}.
A contingency table is a table in which all possible events (or outcomes) for one variable are listed as
row headings, all possible events for a second variable are listed as column headings, and the value entered in
each cell of the table is the frequency of each joint occurrence.


\begin{center}
\begin{tabular}{|c||c|c||c|}
  \hline
  % after \\: \hline or \cline{col1-col2} \cline{col3-col4} ...
    & C & E & Total \\ \hline \hline
  M & 44 & 18 & 62 \\ \hline
  F & 16 & 22 & 38 \\ \hline \hline
  Total & 60 & 40 & 100 \\ \hline
\end{tabular}
\end{center}

}
%-------------------------------------------------------%
\frame{
\frametitle{Contingency Tables}
It is now easy to deduce the probabilities of the respective events, by looking at the totals for each row and column.
\begin{itemize}
\item P(C) = 60/100 = 0.60
\item P(E) = 40/100 = 0.40
\item P(M) = 62/100 = 0.62
\item P(F) = 38/100 = 0.38
\end{itemize}
\textbf{Remark:}\\
The information we were originally given can also be expressed as:
\begin{itemize}
\item $P(C \cap M) = 44/100 = 0.44$
\item $P(C \cap F) = 16/100 = 0.16$
\item $P(E \cap M) = 18/100 = 0.18$
\item $P(E \cap F) = 22/100 = 0.22$
\end{itemize}
}
%-------------------------------------------------------%
\frame{
\frametitle{Joint Probability Tables}

A \textbf{\emph{joint probability table}} is similar to a contingency table, but for that the value entered in
each cell of the table is the probability of each joint occurrence. Often, the probabilities in such a table are based
on observed frequencies of occurrence for the various joint events.
\begin{center}
\begin{tabular}{|c||c|c||c|}
  \hline
  % after \\: \hline or \cline{col1-col2} \cline{col3-col4} ...
    & C & E & Total \\ \hline \hline
  M & 0.44 & 0.18 & 0.62 \\ \hline
  F & 0.16 & 0.22 & 0.38 \\ \hline \hline
  Total & 0.60 & 0.40 & 1.00 \\ \hline
\end{tabular}
\end{center}
}
%-------------------------------------------------------%
\frame{
\frametitle{Marginal Probabilities}
\begin{itemize}
\item In the context of joint probability tables, a  \textbf{\emph{marginal probability}} is so named because it is a marginal total of
a row or a column. \item Whereas the probability values in the cells of the table are probabilities of joint occurrence, the marginal
probabilities are the simple (i.e. unconditional) probabilities of particular events.
\item From the first year intake example, the marginal probabilities are $P(C)$, $P(E)$, $P(M)$ and $P(F)$ respectively.
\end{itemize}

}
%-------------------------------------------------------%
\frame{
\frametitle{Conditional Probabilities : Example 1}

Recall the definition of conditional probability:
\[ P(A|B) = \frac{P(A \cap B)}{P(B)} \]

Using this formula, compute the following:
\begin{enumerate}
\item $P(C|M)$ : Probability that a student is a computer science student, given that he is male.
\item $P(E|M)$ : Probability that a student studies engineering, given that he is male.
\item $P(F|E)$ : Probability that a student is female, given that she studies engineering.
\item $P(E|F)$ : Probability that a student studies engineering, given that she is female.
\end{enumerate}
Refer back to the contingency table to appraise your results.
}
%-------------------------------------------------------%
\frame{
\frametitle{Conditional Probabilities : Example 1}

\textbf{Part 1)} Probability that a student is a computer science student, given that he is male.
\[ P(C|M) = \frac{P(C \cap M)}{P(M)}  = \frac{0.44}{0.62} = 0.71 \]
\textbf{Part 2)} Probability that a student studies engineering, given that he is male.
\[ P(E|M) = \frac{P(E \cap M)}{P(M)}  = \frac{0.18}{0.62} = 0.29 \]

}

%-------------------------------------------------------%
\frame{
\frametitle{Conditional Probabilities : Example 1}

\textbf{Part 3)} Probability that a student is female, given that she studies engineering.
\[ P(F|E) = \frac{P(F \cap E)}{P(E)}  = \frac{0.22}{0.40} = 0.55 \]

\textbf{Part 4)} Probability that a student studies engineering, given that she is female.
\[ P(E|F) = \frac{P(E \cap F)}{P(F)}  = \frac{0.22}{0.38} = 0.58 \]


Remark: $P(E \cap F)$ is the same as $P(F \cap E)$.


}

\end{document}

%--------------------------------------------------------------------------------%
\frame{
\frametitle{R Statistical Computing}
\begin{itemize} \item
\texttt{R} is a computing software for statistical analysis \item The package is available for all popular operating systems: Windows , Mac OS and Linux
\item It is free!
\item Everyone (knowledgeable enough) can contribute to the software by
writing a package. Packages are available for download through a convenient facility
\item \texttt{R} is fairly well documented and the documentation is available either
from the program help menu or from the web-site.
\item \texttt{R} is the top choice of statistical software among academic statisticians
but also very popular in industry.
\item \texttt{R} is a powerful tool not only for doing statistics but also all kind of
scientific programming.
\end{itemize}
}
%--------------------------------------------------------------------------------%
\frame{
\frametitle{R Statistical Computing}
\texttt{R} is a language and environment for statistical computing and graphics.
\texttt{R} provides a wide variety of statistical and graphical techniques, and is highly extensible. Among its tools
one can find implemented
\begin{itemize}
\item linear and nonlinear modelling,
\item classical statistical tests,
\item time-series analysis,
\item classification,
\item clustering,
\item ...and many more.
\end{itemize}
One of \texttt{R}'s strengths is the ease with which well-designed publication quality plots can be produced.
including mathematical symbols and formulae where needed.
}
%--------------------------------------------------------------------------------%
\frame{
\frametitle{R Statistical Computing}

\texttt{R} is an integrated suite of software facilities for data manipulation, calculation and graphical display. It
includes
\begin{itemize}
\item an effective data handling and storage facility,
\item a suite of operators for calculations on arrays, in particular matrices,
\item a large, coherent. integrated collection of intermediate tools for data analysis,
graphical facilities for data analysis and display either on-screen or on hard-copy, and
\item a well-developed, simple and effective programming language which includes conditionals, loops,
user-defined recursive functions and input and output facilities.
\end{itemize}
}
%--------------------------------------------------------------------------------%


\frame{
\frametitle{R Statistical Computing}
Downloading and Installing \texttt{R}:

\begin{itemize}
\item \texttt{R} can be downloaded from the CRAN website: http://cran.r-project.org/
\item You may choose versions for windows, mac and linux.
\item As per the instructions on the respective pages, you require the ``base" distribution.
\item Now you can download the installer for latest version of \texttt{R} , version 2.17.
\item Select the default settings. Once you finish, the \texttt{R} icon should appear on your desktop.
\item Clicking on this icon will start up the program.
\end{itemize}
}
%--------------------------------------------------------------------------------%

\frame{
\frametitle{Next Class}
\begin{itemize} 
\item Continue with probability theory
\item Look at some more worked examples
\item Introduce the concept random variables
\end{itemize}
}
\end{document}










\frame{
\frametitle{The Monty Hall Problem}
\begin{itemize} \item Three cups. One contains the prize, the other two are empty
\item Choose one of the cups
\item What is the probability you have selected the right cup?
\end{itemize}

}
%--------------------------------------------------------------------------------%
\frame{
\frametitle{The Monty Hall Problem}
\begin{itemize} \item I am going to remove one of the empty cups, but not the one just selected.
\item There are two cups on the table, one with the prize, and one empty.
\item I offer you the choice to stick with your choice, or to switch you choice to the other remaining cup.
\item What is the best strategy here?
\end{itemize}
}
%--------------------------------------------------------------------------------%
\frame{
\frametitle{The Monty Hall Problem}
Strategies
\begin{itemize} \item It is better to stick with what you have chosen.
\item It is better to switch to the other cup.
\item It is doesn't matter.
\end{itemize}
}
%--------------------------------------------------------------------------------%
\frame{
\frametitle{The Monty Hall Problem}

The best strategy is to switch.
}

