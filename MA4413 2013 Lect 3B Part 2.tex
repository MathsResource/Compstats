\documentclass[a4]{beamer}
\usepackage{amssymb}
\usepackage{graphicx}
\usepackage{subfigure}
\usepackage{newlfont}
\usepackage{amsmath,amsthm,amsfonts}
%\usepackage{beamerthemesplit}
\usepackage{pgf,pgfarrows,pgfnodes,pgfautomata,pgfheaps,pgfshade}
\usepackage{mathptmx}  % Font Family
\usepackage{helvet}   % Font Family
\usepackage{color}


\setbeamercovered{dynamic}

\title[MA4413]{Statistics for Computing \\ {\normalsize MA4413 Lecture 3X}}
\author[Kevin O'Brien]{Kevin O'Brien \\ {\scriptsize Kevin.obrien@ul.ie}}
\date{Autumn Semester 2013}
\institute[Maths \& Stats]{Dept. of Mathematics \& Statistics, \\ University \textit{of} Limerick}

\renewcommand{\arraystretch}{1.5}

\begin{document}

\begin{frame}
\titlepage
\end{frame}

% Cumulative Distribution Function - Definition (probability mass function)
% Changing the unit space.
% Poisson Example (MA4102)
% Key Rules
% Distinguishing between binomial and poisson.
% Poisson Approximation of Binomial
%

% Last class : Cumulative Tables tables
% Formal Definition
% Sample Space and Partitioning




%---------------------------------------------------------------------%
\begin{frame}
\frametitle{Today's Class}
\begin{itemize}
\item Definition of Cumulative Distribution Function.
\item Binomial Example
\item Using cumulative tables.
\item Poisson distribution - example
%\item Poisson approximation of the binomial Distribution
%\item Poisson approximation - example

\end{itemize}
\end{frame}



%---------------------------------------------------------------------%
\begin{frame}
\frametitle{The Cumulative Distribution Function}
\begin{itemize}
\item The Cumulative Distribution Function, denoted $F(x)$, is a common way that the probabilities
of a random variable (both discrete and continuous) can be summarized.
\item The Cumulative Distribution Function, which also can be
described by a formula or summarized in a table, is defined as:
\[F(x) = P(X \leq x) \]
\item The notation for a cumulative distribution function, F(x), entails using a capital
"F".  (The notation for a probability mass or density function, f(x), i.e. using a lowercase "f". The notation is not interchangeable.
\end{itemize}
\end{frame}

%---------------------------------------------------------------------%
\begin{frame}
\frametitle{Useful Results}
(Demonstration on the blackboard re: partitioning of the sample space, using examples on next slide)
\begin{itemize}
\item $P(X \leq 1) = P(X=0) + P(X=1)$
\item $P(X \leq r) = P(X=0)+ P(X=1) + \ldots P(X= r)$
\item $P(X \leq 0) = P(X=0)$
\item $P(X = r) = P(X \geq r ) - P(X \geq r + 1)$
\item \textbf{Complement Rule}: $P(X \leq r-1) = P(X < r) = 1 - P(X \geq r)$
\item \textbf{Interval Rule}:$ P(a \leq X \leq  b)= P(X \geq a) - P(X \geq b + 1).$
\end{itemize}
For the binomial distribution, if the probability of success is greater than 0.5, instead of
considering the number of successes, to use the table we consider
the number of failures.
\end{frame}
%---------------------------------------------------------------------%


%---------------------------------------------------------------------%
\begin{frame}
\frametitle{Binomial Example 1}
Suppose a signal of 100 bits is transmitted and the probability of
sending a bit correctly is 0.9. What is the probability of
\begin{enumerate}
\item at least 10 errors
\item exactly 7 errors
\item Between 5 and 15 errors (inclusively).
\end{enumerate}
\end{frame}
%---------------------------------------------------------------------%
\begin{frame}
\frametitle{Binomial Example 1}
\begin{itemize}
\item Since the probability of success is 0.9. We consider the distribution
of the number of failures (errors).
\item We reverse the definition of `success' and `failure'. Success is now defined as an error.
\item The probability that a bit is sent incorrectly is 0.1.
\item Let X be the total number of errors. $X \sim B(100, 0.1)$.
\item Answer : $P(X \geq 10) = 0.5487$.
\item $P(X = 7)=P(X \geq 7) - P(X \geq 8) =0.8828 - 0.7939 = 0.0889$.
\item $P(5 \leq X  \leq 15) = P(X \geq 5) - P(X \geq 16) =0.9763 - 0.0399 = 0.9364$
\end{itemize}
\end{frame}

%---------------------------------------------------------------------------%
\frame{
\frametitle{The Poisson Probability Distribution}
\begin{itemize}
\item A Poisson random variable is the number of successes that result from a Poisson experiment.
\item The probability distribution of a Poisson random variable is called a Poisson distribution.
\item Very Important: This distribution describes the number of occurrences in a \textbf{\emph{unit period (or space)}}
\item Very Important: The expected number of occurrences is $m$
\end{itemize}
}
\begin{frame}
\frametitle{The Poisson Probability Distribution}
We use the following notation.
\[X \sim Poisson(m) \]
Note the expected number of occurrences per unit time is conventionally denoted $\lambda$ rather than $m$.
\bigskip
As the Murdoch Barnes cumulative Poisson Tables (Table 2) use $m$, so shall we. Recall that Tables 2 gives values of the probability $P(X \geq r )$, when X has a Poisson distribution with
parameter $m$.

\end{frame}
%---------------------------------------------------------------------%
%---------------------------------------------------------------------------%
\begin{frame}
\frametitle{The Poisson Probability Distribution}
Consider cars passing a point on a rarely used country road. Is this a Poisson Random Variable?
Suppose
\begin{enumerate}
\item Arrivals occur at an average rate of $m$ cars per unit time.
\item The probability of an arrival in an interval of length k
is constant.
\item The number of arrivals in two non-overlapping
intervals of time are independent.
\end{enumerate}
This would be an appropriate use of the Poisson Distribution.
\end{frame}

%---------------------------------------------------------------------%
\begin{frame}
\frametitle{Changing the unit time.}

\begin{itemize}
\item The number of arrivals, X, in an interval of length $t$ has a
Poisson distribution with parameter $\mu = mt$.
\item $m$ is the expected number of arrivals in a unit time period.
\item $\mu$ is the expected number of arrivals in a time period $t$, that is different from the unit time period.
\item Put simply : if we change the time period in question, we adjust the Poisson mean accordingly.
\item If 10 occurrences are expected in 1 hour, then 5 are expected in 30 minutes. Likewise, 20 occurrences are expected in 2 hours, and so on.
\item (Remark : we will not use $\mu$ in this context anymore).
\end{itemize}
\end{frame}


%---------------------------------------------------------------------%
\begin{frame}
\frametitle{Poisson Example}
A motor dealership which specializes in agricultural machinery sells one vehicle every 2 days, on average. Answer the following questions.
    \begin{enumerate}
    \item  What is the probability that the dealership sells at least one vehicle in one particular day?
    \item  What is the probability that the dealership will sell exactly one vehicle in one particular day?
    \item  What is the probability that the dealership will sell 4 vehicles or more in a six day working week?
    \end{enumerate}
\end{frame}

%---------------------------------------------------------------------%
\begin{frame}
\frametitle{Poisson Example}

    \begin{enumerate}
    \item Expected Occurrences per Day: m = 0.5
    \item Probability that the dealership sells at least one vehicle in one particular day?
    \[ P(X \geq 1) = 0.3935 \]
    \item Probability that the dealership will sell exactly one vehicle in one particular day?
    \[ P(X = 1) = P(X \geq 1) - P(X \geq 2)  = 0.3935 - 0.0902 = 0.3031\]
    \item Probability that the dealership will sell 4 vehicles or more in a six day working week?
    \begin{itemize}
    \item For a 6 day week, m=3
    \item $P(X \geq 4) = 0.3528$
    \end{itemize}
    \end{enumerate}
\end{frame}

%---------------------------------------------------------------------%
\begin{frame}
\frametitle{Knowing which distribution to use}
\begin{itemize}
\item For the end of semester examination, you will be required to know when it is appropriate to use the Poisson distribution, and when to use the binomial distribution.
\item Recall the key parameters of each distribution.
\item Binomial : number of \textbf{\emph{successes}} in $n$ \textbf{\emph{independent trials}}.
\item Poisson : number of \textbf{\emph{occurrences}} in a \textbf{\emph{unit space}}.
\end{itemize}
\end{frame}

\end{document}
%---------------------------------------------------------------------%
\begin{frame}
\frametitle{Poisson Approximation of the Binomial}

\begin{itemize}
\item The Poisson distribution can sometimes be used to approximate the
binomial distribution
\item When the number of observations n is large, and the success probability
p is small, the $Bin(n,p)$ distribution approaches the Poisson distribution
with the parameter given by $m = np$.
\item This is useful since the computations involved in calculating binomial
probabilities are greatly reduced.
\item As a rule of thumb, n should be greater than 50 with p very small, such
that np should be less than 5.
\item If the value of p is very high, the definition of what constitutes a
``success" or ``failure" can be switched.
\end{itemize}
\end{frame}

%---------------------------------------------------------------------%
\begin{frame}
\frametitle{Poisson Approximation: Example}

\begin{itemize}
\item Suppose we sample 1000 items from a production line that is producing, on
average, $0.1\%$ defective components.
\item Using the binomial distribution, the probability of exactly 3 defective items in
our sample is
\[P(X = 3) = ^{1000}C_{3} \times 0.001^{3} \times 0.999^{997}\]
\end{itemize}
\end{frame}

%---------------------------------------------------------------------%
\begin{frame}
\frametitle{Poisson Approximation: Example}
Lets compute each of the component terms individually.

\begin{itemize}
\item $^{1000}C_{3}$
\[^{1000}C_{3} = \frac{1000 \times 999 \times 998}{3 \times 2 \times 1} = 166,167,000\]
\item $0.001^3$
\[0.001^3 = 0.000000001\]
\item $0.999^{997}$
\[0.999^{997} = 0.36880\]
\end{itemize}


Multiply these three values to compute the binomial probability
$P(X = 3) = 0.06128$
\end{frame}

\begin{frame}
\frametitle{Poisson Approximation: Example}
\begin{itemize}
\item Lets use the Poisson distribution to approximate a solution.
\item First check that $n \geq 50$ and $np < 5$ (Yes to both).
\item We choose as our parameter value $m = np = 1000 \times 0.001 = 1$
\end{itemize}
\[P(X = 3) = \frac{e^{-1} \times 1^3}{3!} = \frac{e^{-1}}{6} = \frac{0.36787}{6} = 0.06131 \]
Compare this answer with the Binomial probability
P(X = 3) = 0.06128.
Very good approximation, with much less computation effort.
\end{frame}
%---------------------------------------------------------------------%
\end{document}
