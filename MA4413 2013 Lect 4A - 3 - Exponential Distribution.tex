\begin{frame}[fragile]
\frametitle{Continuous Distributions: Current Status}
\begin{itemize}
\item (The Continuous Uniform Distribution, Not examinable)
\item The Exponential Distribution (Examinable for midterm)
\item (Exponential Distribution is the Cut-off point for Mid-Term 1)
\item The Normal Distribution
\item The Standard Normal (Z) Distribution.
\item Applications of Normal Distribution
\end{itemize}
\end{frame}

\begin{frame}[fragile]
\frametitle{Exponential Distribution}
The Exponential Distribution may be used to answer the following questions:
\begin{itemize}
\item How much time will elapse before an earthquake occurs in a given region?
\item How long do we need to wait before a customer enters our shop?
\item How long will it take before a call center receives the next phone call?
\item How long will a piece of machinery work without breaking down?
\end{itemize}
\end{frame}

\begin{frame}[fragile]
\frametitle{Exponential Distribution}

\begin{itemize}
\item All these questions concern the time we need to wait before a given event occurs. If this waiting time is unknown, it is often appropriate to think of it as a random variable having an exponential distribution.
\item Roughly speaking, the time $X$ we need to wait before an event occurs has an exponential distribution if the probability that the event occurs during a certain time interval is proportional to the length of that time interval.

\end{itemize}
\end{frame}

%------------------------------------------------------------------------%
\begin{frame}[fragile]
\frametitle{Probability density function}
The probability density function (PDF) of an exponential distribution is

\[
f(x) = \begin{cases}
\lambda e^{-\lambda x}, & x \ge 0, \\
0, & x < 0.
\end{cases}\]
The parameter $\lambda$  is called \textbf{\emph{rate}} parameter.
\end{frame}

%------------------------------------------------------------------------%
\begin{frame}[fragile]
\frametitle{Cumulative density function}
The cumulative distribution function (CDF) of an exponential distribution is

\[
P(X \leq x) = F(x) = \begin{cases}
1-e^{-\lambda x}, & x \ge 0, \\
0, & x < 0.
\end{cases}\]


The complemeent of the CDF (i.e. $P(X \geq x)$ is

\[
P(X \geq x) = \begin{cases}
e^{-\lambda x}, & x \ge 0, \\
0, & x < 0.
\end{cases}\]
\end{frame}

%------------------------------------------------------------------------%
\begin{frame}[fragile]
\frametitle{Expected Value and Variance}
The expected value of an exponential random variable $X$ is:

\[
E[X] = \frac{1}{\lambda}\]
The variance of an exponential random variable $X$ is:

\[
V[X] = \frac{1}{\lambda^2}\]

\end{frame}

%------------------------------------------------------------------------%
\begin{frame}[fragile]
\frametitle{Exponential Distribution: Example}
Assume that the length of a phone call in minutes is an exponential random variable $X$ with parameter
$\lambda = 1/10$. If someone arrives at a phone booth just before you arrive, find the probability that you
will have to wait \begin{itemize}
\item[(a)] less than 5 minutes,
\item[(b)] between 5 and 10 minutes.
\end{itemize}
\end{frame}



%%------------------------------------------------------------------------%
%\begin{frame}[fragile]
%\frametitle{Exponential Distribution: Example}
%\begin{verbatim}
%> dexp(0:10,rate=0.10)
% [1] 0.10000000 0.09048374 0.08187308 0.07408182 0.06703200 0.06065307
% [7] 0.05488116 0.04965853 0.04493290 0.04065697 0.03678794
%>
%> pexp(0:10,rate=0.10)
% [1] 0.00000000 0.09516258 0.18126925 0.25918178 0.32967995 0.39346934
% [7] 0.45118836 0.50341470 0.55067104 0.59343034 0.63212056
%\end{verbatim}
%\end{frame}

%------------------------------------------------------------------------%
\begin{frame}[fragile]
\frametitle{Exponential Distribution: Example}

\begin{itemize}
\item[(a)] $P(X \leq 5)$ = 0.39346934
\item[(b)] $P(5 \leq X \leq 10)$ \\ = $P( X \leq 10) - P( X \leq 5)$ \\ = 0.6321- 0.3934 \\ = 0.2386 \\= 23.86 $\%$
\item[(c)] Alternative approach to (b)\\$P(5 \leq X \leq 10)$ \\ = $P( X \geq 5) - P( X \geq 10)$ \\
= $e^{-0.5} - e^{-1}$
=0.6065 - 0.3678\\
= 0.2386 = 23.86 $\%$
\end{itemize}

\end{frame}



%------------------------------------------------------------------------%
\begin{frame}[fragile]
\frametitle{Exponential Distribution}
\begin{itemize}
\item The Exponential Rate
\item Related to the Poisson mean (m)
\item If we expect 12 occurrences per hour - what is the rate?
\item We would expected to wait 5 minutes between occurrences.
\end{itemize}
\end{frame}
