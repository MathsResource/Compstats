 \documentclass[a4paper,12pt]{article}
%%%%%%%%%%%%%%%%%%%%%%%%%%%%%%%%%%%%%%%%%%%%%%%%%%%%%%%%%%%%%%%%%%%%%%%%%%%%%%%%%%%%%%%%%%%%%%%%%%%%%%%%%%%%%%%%%%%%%%%%%%%%%%%%%%%%%%%%%%%%%%%%%%%%%%%%%%%%%%%%%%%%%%%%%%%%%%%%%%%%%%%%%%%%%%%%%%%%%%%%%%%%%%%%%%%%%%%%%%%%%%%%%%%%%%%%%%%%%%%%%%%%%%%%%%%%
\usepackage{eurosym}
\usepackage{vmargin}
\usepackage{amsmath}
\usepackage{multicol}
\usepackage{graphics}
\usepackage{epsfig}
\usepackage{framed}
\usepackage{subfigure}
\usepackage{fancyhdr}

\setcounter{MaxMatrixCols}{10}
%TCIDATA{OutputFilter=LATEX.DLL}
%TCIDATA{Version=5.00.0.2570}
%TCIDATA{<META NAME="SaveForMode" CONTENT="1">}
%TCIDATA{LastRevised=Wednesday, February 23, 2011 13:24:34}
%TCIDATA{<META NAME="GraphicsSave" CONTENT="32">}
%TCIDATA{Language=American English}

%\pagestyle{fancy}
\setmarginsrb{20mm}{0mm}{20mm}{25mm}{12mm}{11mm}{0mm}{11mm}
%\lhead{MA4413 2013} \rhead{Mr. Kevin O'Brien}
%\chead{Midterm Assessment 1 }
%\input{tcilatex}

\begin{document}





%---------------------------------------------------------------- %
\newpage
\noindent {\Large \textbf{MA4413 Weeks 10 and 11 Tutorials}}
\section*{Question 1 (Paired t-test)}
The weight of 10 students was observed before commencement of their studies and after graduation (in kgs). By calculating the realisation of the appropriate test statistic, test the hypothesis that the mean weight of students increases during their studies at a significance level of  5\%. 
%--------------------------------%
\begin{center}
\begin{tabular}{|c|c|c|c|c|c|c|c|c|c|c|}
\hline
Student	&	1	&	2	&	3	&	4	&	5	&	6	&	7	&	8	&	9	&	10	\\ \hline
Weight before	&	68	&	74	&	59	&	65	&	82	&	67	&	57	&	90	&	74	&	77	\\ \hline
Weight after	&	71	&	73	&	61	&	67	&	85	&	66	&	61	&	89	&	77	&	83	\\ \hline
\end{tabular} 
\end{center}

%-------------------------------------------%
\noindent \textbf{[Recall Descriptive Statistics]}\\
\noindent You may be required to carry out these calculations in the exam.
\begin{itemize}
\item Case-wise differences are 
\[ d = \{3, -1,  2,  2,  3, -1,  4, -1,  3,  6  \}\]
\item The sum of case-wises differences and squared case-wise differences are $\sum d_i = 20$ and $\sum d_i^2 = 90$ respectively.
\item Mean of case-wise differences $\bar{d}=2.00$.
\[ \bar{d} = \frac{3 + (-1) + 2 + \ldots + 6}{10} = \frac{20}{10} \]
\item Standard deviation of casewise differences $s_d= 2.36 $\\
(Modified version of standard deviation formula)


\[ s_d = \sqrt{\frac{ \sum(d^2_i) - \frac{(\sum d_i)^2}{n}}{n-1}}\]
\[ s_d = \sqrt{\frac{ 90 - \frac{(20)^2}{10}}{9}} = \sqrt{\frac{50}{9}} = 2.36 \]

\item Standard Error
\[ S.E.(d) = \frac{s_d}{\sqrt{n}} =\frac{2.36}{\sqrt{10}} = 0.745\]
\item From Murdoch Barnes, the CV is 1.812 (small sample,df = 9,one-tailed procedure)
\end{itemize}


\noindent \textbf{Writing the Hypotheses}
\begin{description}
\item[$H_0$] $\mu_d \leq 0$ \\mean of case-wise differences not a positive number. (i.e. no increase in weight)
\item[$H_1$] $\mu_d > 0$ \\mean of case-wise differences is a positive number. (i.e. increase in weight)
\end{description}
%-------------------------------------------%


%--------------------------------%
\subsection*{Question 1 Part B}
Calculate a 95\% confidence interval for the amount of weight that students put on during their studies. Using this confidence interval, test the hypotheses that on average students put on \textbf{3 kilos} during their studies
%ii) students lose 3 kilos during their studies.

%(What assumption was made in order to both carry out the test and calculate the confidence interval?)

\section*{Question 2 (Two Sample Means - One Tailed)}
%good
A pharmaceutical company wants to test, a new medication for blood pressure. Tests
for such products often include a `\textit{treatment group}' of people who use the drug and a `\textit{control group}'of people who did not use the drug. 50 people with high blood pressure are given the new drug and 100 others, also with high blood pressure, are not given the drug. 

The systolic blood pressure is measured for each subject, and the sample statistics
are given below. Using a 0.05 level of significance, test the claim that the new drug \textbf{reduces}
blood pressure. 
%Would you recommend advertising that the new drug does not aect blood pressure?
\begin{center}
\begin{tabular}{|c||c|}
\hline 
Treatment & Control \\ \hline \hline
$n_1$ = 50 & $n_2$ = 100 \\ \hline
$x_1$ = 189.4  & $x_1$ = 203.4  \\ \hline
$s_1$ = 39.0 & $s_1$ = 39.4 \\ \hline
\end{tabular} 
\end{center}
\textbf{\textit{Standard Error Formula}}
\[ S.E.(\bar{x}_1 - \bar{x}_2)  = \sqrt{\frac{s_1^2}{n_1} + \frac{s_2^2}{n_2}} \]
%----------------------------------------------------------------------- %
\section*{Question 3 (Two Sample Means - One Tailed)}
The average mass of a sample of 64 Irish teenagers (Let say - 18 year old males) was 73.5kg with a variance of 100kg$^2$. 
The average mass of an equivalent sample of 81 Japanese teenagers was 68.5kg with a variance of 81 kg$^2$. 
\begin{itemize}
\item[(i)] Test the hypothesis that Irish students are larger (in terms of mass) than Japanese teenagers.
%\item[(i)] By calculating the appropriate $p-$value, test the null hypothesis that the mean mass of 
%all Irish students is 70kg at significance levels of 5\%. 
%\item[(ii)] Using the appropriate confidence interval, test the hypotheses that the average mass of all Irish students is a) 70kg, b) 72kg, c) 75kg at a significance level of 5\%.
\end{itemize}

\textbf{\textit{Standard Error Formula}}
\[ S.E.(\bar{x}_1 - \bar{x}_2)  = \sqrt{\frac{s_1^2}{n_1} + \frac{s_2^2}{n_2}} \]
\[ S.E.(\bar{x}_1 - \bar{x}_2)  = \sqrt{\frac{10^2}{64} + \frac{9^2}{81}}  = \sqrt{2.56} = 1.6\]
\newpage


%-------------------------------------%
\section*{Question 4 (Two Sample Means, small samples, one tailed)}
The working lifetimes of 100 of both of two different types of batteries were observed. The mean lifetime for the sample of type 1 batteries was 25 hrs with a standard deviation of 4hrs. The mean lifetime for the sample of type 1 batteries was 23 hrs with a standard deviation of 3hrs. 
\begin{center}
\begin{tabular}{|c||c|}
\hline 
Type 1 & Type 2 \\ \hline \hline
$n_1$ = 100 & $n_2$ = 100 \\ \hline
$x_1$ = 25 hours & $x_1$ = 23 hours \\ \hline
$s_1$ = 4 hours & $s_1$ = 3 hours \\ \hline
\end{tabular} 
\end{center}
\begin{itemize}
\item[(i)] Test the hypothesis that the mean working lifetimes of these batteries do not differ at a significance level of 5\% .

\item[(ii)] Calculate a 95\% confidence interval for the difference between the average working lifetimes of these batteries. 
\item[(iii)] Using this confidence interval, test the hypothesis that battery 1 on average works for 3 hours longer than battery 2.
\end{itemize}


%----------------------------------------------------------------------- %
\section*{Question 5 (Two Sample proportions, one tailed)}

A simple random sample of front-seat occupants involved in car crashes were taken. 
The first sample was on cars with airbags available and it was found that there were 29 occupant fatalities out of a total of 1110 occupants. The second sample was on cars with no airbags available and
there were 62 fatalities out of a total 1553 occupants.
\begin{itemize}
\item[(i)] Using a 5\% significance level, determine whether or not there is a difference in the proportion of fatality rates of occupants in cars with airbags and cars without airbags.
\item[(ii)] Calculate a 95\% confidence interval for the difference between the two proportions of fatality rates.
\end{itemize}

\noindent \textbf{\textit{Standard Error Formula }}\\
Confidence Intervals
\[ S.E.(\hat{p}_1 - \hat{p}_2)  = \sqrt{\frac{\hat{p}_1 \times (100 - \hat{p}_1)}{n_1} + \frac{\hat{p}_2 \times (100 - \hat{p}_2)}{n_2}} \]
Hypothesis testing
\[ S.E.(\pi_1 - \pi_2)  = \sqrt{\bar{p} \times (100 - \bar{p}) \times \left( \frac{1}{n_1} + \frac{1}{n_2}\right)} \]
Aggregate Sample Proportion
\[  \bar{p} = \frac{x_1+x_2}{n_1+n_2} \]


\noindent \textbf{\textit{Confidence Intervals (in terms of percentages) }}\\
95\% confidence interval
\[ (\hat{p}_1 - \hat{p}_2 ) \times (1.96 \times S.E.(\hat{p}_1 - \hat{p}_2))\]
\[ 1.4 \times (1.96 \times 0.683) =  (1.27,1.53)\]
%----------------------------------------------------------------------- %
\section*{Question 6 (Two Sample proportions, one tailed)}
\begin{itemize}
\item The government wishes to increase the proportion of people taking government training courses who obtain a job in the following 3 months. \item Before they introduced the new schemes this figure was 58\%, according to sample of 400 people, with 232 successes. \item A survey of 300 people who took the new courses indicated that 188 of them gained a job. A government official stated that this indicates that the new courses have been more successful. \item Is this statement reasonable at a significance level of 5\%?
\end{itemize}

\noindent{\textbf{Some Calculations}}

\begin{itemize}
\item Aggregate proportion
\[ \bar{p} = \frac{232 + 188}{400+300} = \frac{420}{700} =60\%\]
\item Standard Error for Hypothesis Test
\[S.E\]\[ S.E.(\pi_1 - \pi_2)  = \sqrt{60 \times 40) \times \left( \frac{1}{400} + \frac{1}{300}\right)}  = 3.74\]

\end{itemize}
\newpage
%-------------------------------------%
\section*{Question 7 - Two Sample Means (Small Samples)}
A new process has been developed to reduce the level of corrosion of car bodies.
\begin{itemize}
\item Experiments were carried out on 11 cars using the new process and 11 cars using the old process. \item The average level of corrosion using the new process was 3.4 with a standard deviation of 0.5. \item The average level of corrosion using the old process was 4.2 with a standard deviation of 0.8. 
\end{itemize} 

\begin{itemize}
\item[(i)] Test the hypothesis that the variance of the level of corrosion does not depend on the process used.
\item[(ii)] Is there any evidence that the new process is better at a significance level of 5\%?
\item[(iii)] Calculate a 95\% confidence interval for the difference between the mean levels of corrosion under the two processes. Can it be stated that the mean level of corrosion is reduced by 1.5 at a significance level of 5\%? 
%\item[(iv)] What assumptions were used in ii) and iii)? 
\end{itemize}

%-------------------------------------%
\section*{Question 8 - Two Sample Means}
Deltatech software has 350 programmers divided into two groups with 200 in Group A
and 150 in Group B. In order to compare the efficiencies of the two groups, the
programmers are observed for 1 day.
%------------------%
\begin{itemize}
\item The 200 programmers in Group A averaged 45.2 lines of code with a standard
deviation of 8.4.
\item The 150 programmers in Group B averaged 42.7 lines of code with a standard
deviation of 5.2.
\end{itemize}
%------------------%
Let $\bar{x}_A$ denote the average number of lines of code per day produced by programmers in
Group A and
let $\bar{x}_A$ be the corresponding statistic for Group B.
Provide an estimate of $\mu_A —\mu_B$ and calculate an approximate 95\% confidence interval for
%------------------%

Test the claim that Group A are more efficient than Group B by
\begin{itemize}
\item[(i)] Interpreting the 95\% confidence interval.
\item[(ii)] Computing the appropriate test statistic.
%\item[(iii)] Computing the appropriate p-value.
\end{itemize}
%----------------------- %
\section*{Question 9 - Two Sample Proportions}
In a recent British election 40.12\% of the voters voted for the Labour party. A survey of 98 people indicated that 49 of them wish to vote for the Labour party. 
\begin{itemize}
\item[(i)] Does this figure indicate that support for the Labour party has changed at a significance level of 5\% (calculate the realisation of the appropriate test statistic)? 
\item[(ii)] Calculate a 95\% confidence interval for the present support of the Labour party. Comment on your result taking your conclusion from part i) into account. 
\end{itemize}


\newpage
\section*{Question 10 - Testing Equality of Variances}
Interpret the output from the following tests of equality of variances. State your conclusion both by referencing the $p-$value and the confidence interval. You may assume the significance level is 5\%.

(Remark : This procedure is a one-tailed procedure. However, we will base our conclusion on whether or not we arbitrarily decide the p-value is large or small )
\begin{framed}
\begin{verbatim}
> var.test(X,Y)

        F test to compare two variances

data:  X and Y
F = ………, num df = 13, denom df = 13, p-value = 0.02725
alternative hypothesis: 
   true ratio of variances is not equal to 1 
95 percent confidence interval:
  1.164437 11.299050 
sample estimates:
ratio of variances 
          ………… 
\end{verbatim}
\end{framed}
\begin{framed}
\begin{verbatim}
> var.test(X,Z)

        F test to compare two variances

data:  X and Z 
F = ………, num df = 13, denom df = 11, p-value = 0.7813
alternative hypothesis: 
   true ratio of variances is not equal to 1 
95 percent confidence interval:
 0.2526643 2.7401535 
sample estimates:
ratio of variances 
         ………………
\end{verbatim}
\end{framed}
%\begin{framed}
%\begin{verbatim}
%> var.test(Y,Z)
%
%        F test to compare two variances
%
%data:  Y and Z
%F = ………, num df = 13, denom df = 11, p-value = 0.01616
%alternative hypothesis: true ratio of variances is not equal to 1 
%95 percent confidence interval:
% 0.06965702 0.75543304 
%sample estimates:
%ratio of variances 
%         …………………
%\end{verbatim}
%\end{framed}
%\end{document}
\newpage

\section*{Question 11 - Shapiro-Wilk Test}
Interpret the output from the three Shapiro-Wilk tests. What is the null and alternative hypotheses? State your conclusion for each of the three tests.
\begin{framed}
\begin{verbatim}
> shapiro.test(X)

        Shapiro-Wilk normality test

data:  X
W = 0.9001, p-value = 0.113
>
\end{verbatim}
\end{framed}
\begin{framed}
\begin{verbatim}
> shapiro.test(Y)

        Shapiro-Wilk normality test

data:  Y 
W = 0.8073, p-value = 0.006145
>
\end{verbatim}
\end{framed}
\begin{framed}
\begin{verbatim}
> shapiro.test(Z)

        Shapiro-Wilk normality test

data:  Z
W = 0.9292, p-value = 0.372
\end{verbatim}
\end{framed}
%------------------------------------------------ %
\newpage
\section*{Question 12 - Classification Metrics}
For each of the following classification tables, calculate the following appraisal metrics.
\begin{multicols}{2}
\begin{itemize}	
\item 	accuracy  
\item 	recall
\item 	precision
\item 	F-measure
\end{itemize}
\end{multicols}
	



\begin{center}
\begin{tabular}{|c|c|c|}
\hline  & Predict Negative & Predict Positive \\ 
\hline Observed Negative &	9500	&	85	\\ 
\hline Observed Positive & 	115	&	300	\\ 
\hline 
\end{tabular} 
\end{center}

\begin{center}
\begin{tabular}{|c|c|c|}
\hline  & Predict Negative & Predict Positive \\ 
\hline Observed Negative &	9700	&	140	\\ 
\hline Observed Positive & 	60	&	100	\\ 
\hline 
\end{tabular} 
\end{center}

\begin{center}
\begin{tabular}{|c|c|c|}
\hline  & Predict Negative & Predict Positive \\ 
\hline Observed Negative &	9530	&	10	\\ 
\hline Observed Positive & 	300	&	160	\\ 
\hline 
\end{tabular} 
\end{center}
			


\end{document}
%------------------------------------- %
\section*{Question 6}

In a study of company salaries, salaries paid by 2 different IT companies were randomly selected.
\begin{itemize}
\item For 40 Deltatech employees the mean is 23,870 and the standard deviation is 2,960. 
\item For 35 Echelon employees , the mean is 22,025 and the standard deviation is 3,065.
\end{itemize} 

At the 0.05 level of significance, test the claim that Deltatech employees earn the same as their Echelon counterparts.
%-------------------------------------- %
\section*{Question 7}
Does it pay to take preparatory courses for standardised tests such as the Comptia Exams? 

Using the sample data in the following table, compute the case-wise differences, the mean of the case-wise differences and the standard deviation of the case wise differences for the following data set.

\begin{center}
\begin{tabular}{|c|c|c|c|c|c|c|c|c|c|c|}
\hline  
Student	&	A	&	B	&	C	&	D	&	E	&	F	&	G	&	H	&	I	&	J	\\ \hline
Score Before	&	700	&	840	&	830	&	860	&	840	&	690	&	830	&	1180	&	930	&	1070	\\ \hline
Score After	&	720	&	840	&	820	&	900	&	870	&	700	&	800	&	1200	&	950	&	1080	\\ \hline
\end{tabular} 
\end{center}
%------------------------------------- %



\section*{Question 8}
The average height of a sample of 16 students was 173cm with a variance of 144cm$^2$. The average height of the Irish population is 169cm. 
\begin{itemize}
\item[(i))] Can it be stated at a significance level of 5\% that students are on average taller than the population as a whole? 
\item[(ii))] What assumption is used to carry out this test? Is this assumption reasonable?
\end{itemize}



%-------------------------------------%
\section*{Question 9}
A coal-fired power plant is considering two different systems for pollution abatement.
The first system has reduced the emission of pollutants to acceptable levels 68\% of the time,
as determined from 200 air samples. The second, more expensive system has reduced the
emission of
pollutants to acceptable levels 70\% of the time, as determined from 250 air samples. lf the
expen sive system is significantly more eilective than the inexpensive system in reducing the
pollutants to acceptable levels, then the management of the power plant will install the
expensive system.

\begin{itemize}
\item[(i)] Which system will be installed if management uses a significance level of 0.05 in making
its decision?
\item[(ii)] Construct a 95\% confidence interval for the difference in the two proportions. Interpret
this interval.
\end{itemize}
%------------------%

\section*{Question 10}
It is generally assumed that older people are more likely to vote for the Conservatives than younger people. In a survey, 160 of 400 people over 40 and 120 of 400 people under 40 stated they would vote Conservative. 
\begin{itemize}
\item[(i)] Do the data support this hypothesis at a significance level of 5\%?
\item[(ii)] Calculate a 95\% confidence interval for the difference between the proportion of people over 40 voting Conservative and the proportion of people below 40 voting Conservative. 
\end{itemize}



\end{document}
